\documentclass[11pt,a4paper]{article}
\usepackage[margin=1in]{geometry}
\usepackage{graphicx}
\usepackage{amsmath}
\usepackage{amssymb}
\usepackage{booktabs}
\usepackage{hyperref}
\usepackage{float}

\title{AI3603 Homework 3: Bayesian Networks\\Risk Factor Analysis}
\author{}
\date{}

\begin{document}

\maketitle

\section{Implementation}

The Bayesian network inference system was implemented in \texttt{BayesianNetworks.py} with the following core functions:

\begin{itemize}
    \item \textbf{joinFactors}: Merges two factor tables by joining on common variables and multiplying probabilities
    \item \textbf{marginalizeFactor}: Marginalizes out a variable by summing probabilities over its values
    \item \textbf{evidenceUpdateNet}: Updates network by filtering factors to match evidence values
    \item \textbf{inference}: Performs variable elimination by joining all factors, marginalizing hidden variables, and normalizing
\end{itemize}

All test cases in \texttt{BayesNetworkTestScript.py} pass successfully, confirming correct implementation.

\section{Network Structure and Size}

The Bayesian network for health risk factors contains 12 variables with the following directed edges:

\begin{itemize}
    \item \textbf{Root}: income $\rightarrow$ smoke, exercise, long\_sit, stay\_up, bmi
    \item \textbf{Health indicators}:
    \begin{itemize}
        \item bmi $\rightarrow$ bp, diabetes
        \item smoke, exercise $\rightarrow$ cholesterol
    \end{itemize}
    \item \textbf{Outcomes}: bp, cholesterol $\rightarrow$ stroke, attack, angina
\end{itemize}

This structure encodes the causal assumptions: income affects lifestyle habits and body mass; habits affect intermediate health indicators; indicators affect disease outcomes.

\textbf{Network size}: 192 probabilities
\begin{itemize}
    \item P(income): 8
    \item P(smoke$|$income), P(exercise$|$income), P(long\_sit$|$income), P(stay\_up$|$income): 4 $\times$ 16 = 64
    \item P(bmi$|$income): 32
    \item P(bp$|$bmi): 16
    \item P(cholesterol$|$smoke,exercise): 8
    \item P(diabetes$|$bmi): 16
    \item P(stroke$|$bp,cholesterol), P(attack$|$bp,cholesterol), P(angina$|$bp,cholesterol): 3 $\times$ 16 = 48
\end{itemize}

\textbf{Full joint distribution}: 131,072 probabilities (8$\times$2$\times$2$\times$2$\times$2$\times$4$\times$4$\times$2$\times$4$\times$2$\times$2$\times$2)

\textbf{Compression ratio}: 682.67$\times$

The factorized representation provides massive computational savings.

\section{Health Outcomes Analysis}

\subsection{Bad vs Good Habits}

Table~\ref{tab:habits} shows outcome probabilities for bad habits (smoking, no exercise, long sitting, staying up late) versus good habits.

\begin{table}[H]
\centering
\begin{tabular}{lrrr}
\toprule
Outcome & Bad Habits & Good Habits & Difference \\
\midrule
Diabetes & 0.1365 & 0.1304 & +0.0061 \\
Stroke & 0.0427 & 0.0387 & +0.0040 \\
Attack & 0.0648 & 0.0562 & +0.0086 \\
Angina & 0.0694 & 0.0587 & +0.0107 \\
\bottomrule
\end{tabular}
\caption{Probability of health outcomes given habits}
\label{tab:habits}
\end{table}

Bad habits increase all health risks, with the largest effect on angina (+1.07\%) and smallest on diabetes (+0.61\%).

\subsection{Poor vs Good Health}

Table~\ref{tab:health} shows outcome probabilities for poor health indicators (high BP, high cholesterol, obese) versus good health.

\begin{table}[H]
\centering
\begin{tabular}{lrrr}
\toprule
Outcome & Poor Health & Good Health & Difference \\
\midrule
Diabetes & 0.2439 & 0.0570 & +0.1869 \\
Stroke & 0.0869 & 0.0143 & +0.0726 \\
Attack & 0.1393 & 0.0189 & +0.1204 \\
Angina & 0.1596 & 0.0154 & +0.1442 \\
\bottomrule
\end{tabular}
\caption{Probability of health outcomes given health status}
\label{tab:health}
\end{table}

Poor health indicators have a much stronger effect than habits, increasing diabetes risk by 18.7\% and angina by 14.4\%.

\begin{figure}[H]
\centering
\includegraphics[width=0.95\textwidth]{assets/habits_health_comparison.png}
\caption{Comparison of habits and health effects on outcomes}
\end{figure}

\section{Income Effect Analysis}

Figure~\ref{fig:income} shows the probability of each health outcome across income levels. All outcomes show a clear negative correlation with income:

\begin{itemize}
    \item \textbf{Diabetes}: Decreases from 14.7\% (income < \$10K) to 12.3\% (income > \$75K)
    \item \textbf{Stroke}: Decreases from 4.16\% to 3.92\%
    \item \textbf{Attack}: Decreases from 6.17\% to 5.78\%
    \item \textbf{Angina}: Decreases from 6.54\% to 6.08\%
\end{itemize}

Higher income is associated with better health outcomes, likely through access to healthcare, healthier lifestyles, and reduced stress.

\begin{figure}[H]
\centering
\includegraphics[width=0.95\textwidth]{assets/income_effect.png}
\caption{Effect of income level on health outcomes}
\label{fig:income}
\end{figure}

\section{Testing Independence Assumptions}

\subsection{Independence Assumption}

The original network has no direct edges from habits (smoke, exercise) to outcomes (diabetes, stroke, attack, angina). This encodes a \textbf{conditional independence assumption}: given the intermediate health indicators (bp, cholesterol, bmi), the outcomes are independent of smoking and exercise habits.

Formally: $P(\text{outcome} | \text{habits}, \text{indicators}) = P(\text{outcome} | \text{indicators})$

This assumes habits only affect outcomes indirectly through their effects on bp, cholesterol, and bmi.

\subsection{Testing the Assumption}

To test this, a second network was created with direct edges from smoking and exercise to all four outcomes. Results are shown in Table~\ref{tab:network2}.

\begin{table}[H]
\centering
\begin{tabular}{lrrrr}
\toprule
Outcome & \multicolumn{2}{c}{Network 1 (Original)} & \multicolumn{2}{c}{Network 2 (Direct Links)} \\
\cmidrule(lr){2-3} \cmidrule(lr){4-5}
 & Bad & Good & Bad & Good \\
\midrule
Diabetes & 0.1365 & 0.1304 & 0.1954 & 0.1013 \\
Stroke & 0.0427 & 0.0387 & 0.0697 & 0.0262 \\
Attack & 0.0648 & 0.0562 & 0.1104 & 0.0334 \\
Angina & 0.0694 & 0.0587 & 0.1067 & 0.0395 \\
\bottomrule
\end{tabular}
\caption{Comparison of networks with and without direct habit links}
\label{tab:network2}
\end{table}

Adding direct edges dramatically increases the difference between bad and good habits:
\begin{itemize}
    \item Diabetes: 0.61\% $\rightarrow$ 9.40\% (15.4$\times$ stronger)
    \item Stroke: 0.40\% $\rightarrow$ 4.35\% (10.9$\times$ stronger)
    \item Attack: 0.86\% $\rightarrow$ 7.70\% (9.0$\times$ stronger)
    \item Angina: 1.07\% $\rightarrow$ 6.71\% (6.3$\times$ stronger)
\end{itemize}

\textbf{Conclusion}: The independence assumption is \textbf{invalid}. Smoking and exercise have direct effects on health outcomes beyond what is mediated by bp, cholesterol, and bmi. The original network significantly underestimates the impact of lifestyle habits.

\begin{figure}[H]
\centering
\includegraphics[width=0.95\textwidth]{assets/network_structures.png}
\caption{Network structure comparison}
\end{figure}

\begin{figure}[H]
\centering
\includegraphics[width=0.95\textwidth]{assets/network_difference.png}
\caption{Impact of adding direct habit-outcome links}
\end{figure}

\section{Outcome Interactions}

\subsection{Independence Between Outcomes}

The previous networks have no edges between the four health outcomes (diabetes, stroke, attack, angina). This encodes another \textbf{conditional independence assumption}: given their common parents (bp, cholesterol, habits), the outcomes are independent of each other.

Formally: $P(\text{stroke} | \text{diabetes}, \text{parents}) = P(\text{stroke} | \text{parents})$

This assumes diabetes doesn't directly cause stroke; they only co-occur due to shared risk factors.

\subsection{Testing Diabetes-Stroke Interaction}

To test whether diabetes and stroke interact, a third network was created adding an edge from diabetes to stroke.

\begin{table}[H]
\centering
\begin{tabular}{lrr}
\toprule
Network & P(stroke=1 $|$ diabetes=1) & P(stroke=1 $|$ diabetes=3) \\
\midrule
Network 2 (no edge) & 0.0458 & 0.0387 \\
Network 3 (with edge) & 0.0772 & 0.0337 \\
\midrule
Difference & +0.0071 & -0.0435 \\
\bottomrule
\end{tabular}
\caption{Effect of diabetes on stroke probability}
\end{table}

Without the direct edge, diabetes increases stroke risk by only 0.71\% (4.58\% vs 3.87\%). With the edge, this increases to 4.35\% (7.72\% vs 3.37\%) — a 6.2$\times$ stronger effect.

\textbf{Conclusion}: The independence assumption between diabetes and stroke is \textbf{invalid}. Diabetes has a direct causal effect on stroke risk beyond their common causes (bp, cholesterol). This aligns with medical knowledge that diabetes damages blood vessels and increases stroke risk.

\section{Code Verification}

The implementation was verified against all test cases in \texttt{BayesNetworkTestScript.py}. The test includes:

\begin{itemize}
    \item Bishop textbook car battery/fuel/gauge network examples
    \item Factor joining with different operation orders (commutativity verification)
    \item Marginalization operations
    \item Evidence update functionality
    \item Full inference with evidence and marginalization
    \item Risk factor network queries
\end{itemize}

Figure~\ref{fig:testoutput} shows the test script output, which matches the expected results in Figure 2 of the assignment specification. All probability values are correct, confirming proper implementation of variable elimination inference.

\begin{figure}[H]
\centering
\includegraphics[width=0.85\textwidth]{assets/test_screenshot.png}
\caption{BayesNetworkTestScript.py output showing correct inference results}
\label{fig:testoutput}
\end{figure}

Figure~\ref{fig:analysis} shows the complete analysis output for all five questions, demonstrating that the implementation correctly handles complex multi-variable queries on the health risk factor dataset.

\begin{figure}[H]
\centering
\includegraphics[width=0.95\textwidth]{assets/analysis.png}
\caption{Complete analysis output for Questions 1-5}
\label{fig:analysis}
\end{figure}

\section{Summary}

The implementation correctly performs Bayesian network inference using variable elimination. Analysis of the health risk data revealed:

\begin{enumerate}
    \item Factorized representation provides 683$\times$ compression over full joint distribution
    \item Poor health indicators have stronger effects than bad habits (18.7\% vs 1.1\% for diabetes)
    \item Higher income correlates with better health outcomes across all conditions
    \item Smoking and exercise have direct effects on outcomes beyond bp/cholesterol/bmi
    \item Diabetes directly increases stroke risk beyond shared risk factors
\end{enumerate}

The independence assumptions in the original network are too strong and underestimate the true causal relationships in health risk factors.

\end{document}
